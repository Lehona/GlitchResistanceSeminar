%
% Sample introduction of your thesis
%

\chapter{Introduction}
Physical access to otherwise secure systems has become much more common due to the widespread use of embedded devices in today's world. This circumstance enables attacks that are impossible in more restricted scenarios and thus, the need for countermeasures against such physical attacks. 

Indeed, glitching a system, i.e., introducing a disturbance such that the execution of the processor is influenced, is one of the many ways attackers may be able to leverage this increased level of access. These hardware-induced faults have already been used to compromise commercially available systems, most notably the XBOX 360\,\cite{free60} and PlayStation 3\,\cite{ps3hyper}. Similar to Side-Channel Analysis \todo{Cite something!}, glitching-based attacks offer a new angle of attack against systems that are running semantically secure software without any conventional vulnerabilities.

In this paper we introduce a formal description of hardware-induced faults and give an overview of \todo{Von anderen Papern vorgeschlagen??} defences suggested by related works. We categorise these countermeasures by whether they are software-only, hardware-only or require a hybrid solution and evaluate them within those categories, based on their quality (error detection rate and latency) and their induced overhead in memory or runtime.

\todo[inline]{Short overview of attack categories (control flow violation vs. faulty computation)}
\todo[inline]{Short overview of defence categories (control flow integrity via signatures; redundant computations)}
Describe the aim of your thesis and motivate this aim. Describe previous work and cite some stuff~\cite{Bay1, Ernst}, . Describe how your thesis is structured...
